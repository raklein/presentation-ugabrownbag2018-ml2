\documentclass[]{article}
\usepackage{lmodern}
\usepackage{amssymb,amsmath}
\usepackage{ifxetex,ifluatex}
\usepackage{fixltx2e} % provides \textsubscript
\ifnum 0\ifxetex 1\fi\ifluatex 1\fi=0 % if pdftex
  \usepackage[T1]{fontenc}
  \usepackage[utf8]{inputenc}
\else % if luatex or xelatex
  \ifxetex
    \usepackage{mathspec}
  \else
    \usepackage{fontspec}
  \fi
  \defaultfontfeatures{Ligatures=TeX,Scale=MatchLowercase}
\fi
% use upquote if available, for straight quotes in verbatim environments
\IfFileExists{upquote.sty}{\usepackage{upquote}}{}
% use microtype if available
\IfFileExists{microtype.sty}{%
\usepackage{microtype}
\UseMicrotypeSet[protrusion]{basicmath} % disable protrusion for tt fonts
}{}
\usepackage[margin=1in]{geometry}
\usepackage{hyperref}
\hypersetup{unicode=true,
            pdftitle={Many Labs 2},
            pdfauthor={Richard Klein  LIP/PC2S   Université Grenoble Alpes},
            pdfborder={0 0 0},
            breaklinks=true}
\urlstyle{same}  % don't use monospace font for urls
\usepackage{longtable,booktabs}
\usepackage{graphicx,grffile}
\makeatletter
\def\maxwidth{\ifdim\Gin@nat@width>\linewidth\linewidth\else\Gin@nat@width\fi}
\def\maxheight{\ifdim\Gin@nat@height>\textheight\textheight\else\Gin@nat@height\fi}
\makeatother
% Scale images if necessary, so that they will not overflow the page
% margins by default, and it is still possible to overwrite the defaults
% using explicit options in \includegraphics[width, height, ...]{}
\setkeys{Gin}{width=\maxwidth,height=\maxheight,keepaspectratio}
\IfFileExists{parskip.sty}{%
\usepackage{parskip}
}{% else
\setlength{\parindent}{0pt}
\setlength{\parskip}{6pt plus 2pt minus 1pt}
}
\setlength{\emergencystretch}{3em}  % prevent overfull lines
\providecommand{\tightlist}{%
  \setlength{\itemsep}{0pt}\setlength{\parskip}{0pt}}
\setcounter{secnumdepth}{0}
% Redefines (sub)paragraphs to behave more like sections
\ifx\paragraph\undefined\else
\let\oldparagraph\paragraph
\renewcommand{\paragraph}[1]{\oldparagraph{#1}\mbox{}}
\fi
\ifx\subparagraph\undefined\else
\let\oldsubparagraph\subparagraph
\renewcommand{\subparagraph}[1]{\oldsubparagraph{#1}\mbox{}}
\fi

%%% Use protect on footnotes to avoid problems with footnotes in titles
\let\rmarkdownfootnote\footnote%
\def\footnote{\protect\rmarkdownfootnote}

%%% Change title format to be more compact
\usepackage{titling}

% Create subtitle command for use in maketitle
\newcommand{\subtitle}[1]{
  \posttitle{
    \begin{center}\large#1\end{center}
    }
}

\setlength{\droptitle}{-2em}

  \title{Many Labs 2}
    \pretitle{\vspace{\droptitle}\centering\huge}
  \posttitle{\par}
  \subtitle{Investigating Variation in Replicability across Sample and Setting}
  \author{Richard Klein LIP/PC2S Université Grenoble Alpes}
    \preauthor{\centering\large\emph}
  \postauthor{\par}
      \predate{\centering\large\emph}
  \postdate{\par}
    \date{2018-12-10 (updated: 2018-12-10)}


\begin{document}
\maketitle

class: center, middle

\section{Many Labs 2}\label{many-labs-2}

\begin{center}\rule{0.5\linewidth}{\linethickness}\end{center}

\section{Cause for Concern}\label{cause-for-concern}

--

\subsection{.center{[}{]}}\label{center}

\subsection{.center{[}{]}}\label{center-1}

.center{[}{]}

\begin{center}\rule{0.5\linewidth}{\linethickness}\end{center}

\section{Flexibility in Data
Analysis}\label{flexibility-in-data-analysis}

.center{[}\url{http://fivethirtyeight.com/features/science-isnt-broken}{]}

.center{[}{]}

\begin{center}\rule{0.5\linewidth}{\linethickness}\end{center}

\section{Flexibility in Data
Analysis}\label{flexibility-in-data-analysis-1}

.center{[}\url{http://fivethirtyeight.com/features/science-isnt-broken}{]}

.center{[}{]}

\begin{center}\rule{0.5\linewidth}{\linethickness}\end{center}

\section{Flexibility in Data
Analysis}\label{flexibility-in-data-analysis-2}

.center{[}\url{http://fivethirtyeight.com/features/science-isnt-broken}{]}

.center{[}{]}

\begin{center}\rule{0.5\linewidth}{\linethickness}\end{center}

\section{Failures to Replicate}\label{failures-to-replicate}

--

\begin{itemize}
\item
  \texttt{Reproducibility\ Project:\ Psychology} (OSC, 2015)\\
\item ~
  \subsection{\textasciitilde{}40/100 studies
  replicated}\label{studies-replicated}
\item
  \texttt{Social\ Sciences\ Replication\ Project} (Camerer et al., 2018)
\item
  13/21 replicated
\item ~
  \subsection{All from Science and
  Nature}\label{all-from-science-and-nature}
\item
  \texttt{Multiple\ large-scale\ Registered\ Reports}
\item
  POPS/AMPPS Registered Replication Reports
\end{itemize}

\begin{center}\rule{0.5\linewidth}{\linethickness}\end{center}

\section{Failures to Replicate}\label{failures-to-replicate-1}

--

\begin{itemize}
\item ~
  \subsection{What we know: Many studies are failing to
  replicate}\label{what-we-know-many-studies-are-failing-to-replicate}
\item ~
  \subsection{Why? Not sure}\label{why-not-sure}
\item ~
  \subsection{Could be false positives}\label{could-be-false-positives}
\item
  Could be many other reasons:
\item
  Moderators (known/unknown)
\item
  Lack of care/expertise
\item
  Sensitivity of effects to sample/context
\end{itemize}

\begin{center}\rule{0.5\linewidth}{\linethickness}\end{center}

\section{Many Labs Projects}\label{many-labs-projects}

--

Each ML project examines a different aspect of replication. Each
question requires data colletion at multiple labs.\\
- \texttt{Many\ Labs\ 1} (Klein et al., 2014)\\
- 10/13 successful replications - Little variation between samples --

\begin{itemize}
\item
  \texttt{Many\ Labs\ 2} (Klein et al., in press)
\item ~
  \subsection{Discussing today}\label{discussing-today}
\item
  \texttt{Many\ Labs\ 3} (Ebersole et al., 2016)
\item
  3/10 successful replications
\item ~
  \subsection{Little variation across
  semester}\label{little-variation-across-semester}
\item
  \texttt{Many\ Labs\ 4} (Klein et al., in prep)
\item
  Terror Management Theory-specific
\item ~
  \subsection{\texorpdfstring{Compare expert replications vs
  ``in-house''
  replications}{Compare expert replications vs in-house replications}}\label{compare-expert-replications-vs-in-house-replications}
\item
  \texttt{Many\ Labs\ 5} (Ebersole et al., in prep)
\item
  Follow-up to Reproducibility Project
\end{itemize}

\begin{center}\rule{0.5\linewidth}{\linethickness}\end{center}

\section{Many Labs 2}\label{many-labs-2-1}

--

\subsection{Like Many Labs 1, but a much stronger
test:}\label{like-many-labs-1-but-a-much-stronger-test}

\begin{itemize}
\item ~
  \subsection{Goal: Replicate many different studies all around the
  world and compare if they vary based on the sample of data
  collection.}\label{goal-replicate-many-different-studies-all-around-the-world-and-compare-if-they-vary-based-on-the-sample-of-data-collection.}
\item
  Replicated 28 studies
\item
  Split across two study ``packages'' due to length
\item
  Computerized in Qualtrics
\item ~
  \subsection{Randomized study order, presented
  back-to-back}\label{randomized-study-order-presented-back-to-back}
\item
  Which studies?
\item
  Structured selection process by committee. Documented: osf.io/8cd4r
\item
  Sought open nominations for studies
\item
  Emphasized impact (citations, etc.), diversity of content, possibility
  for variability across sites
\item
  But substantial practical constraints: Short, able to be computerized
\item
  Authors could decline to be replicated
\end{itemize}

\begin{center}\rule{0.5\linewidth}{\linethickness}\end{center}

\section{Many Labs 2}\label{many-labs-2-2}

\begin{longtable}[]{@{}r@{}}
\toprule
\begin{minipage}[b]{0.03\columnwidth}\raggedleft\strut
- Registered Replication Report at AMPPS: - Each study reviewed and
approved by original authors or other experts - Analysis plan(s)
specified in advance (osf.io/c97pd/) - Open data and materials\strut
\end{minipage}\tabularnewline
\midrule
\endhead
\begin{minipage}[t]{0.03\columnwidth}\raggedleft\strut
- Administer packages across 125 samples - Slate 1: 13 studies
administered in each of 61 labs - Slate 2: 15 studies administered in
each of 64 labs - Sites (mostly) randomly assigned to slates - Minimum
of 80 participants per site - 15,305 participants total - Much more
diverse\strut
\end{minipage}\tabularnewline
\bottomrule
\end{longtable}

\section{Many Labs 1 map}\label{many-labs-1-map}

.center{[}{]}

\begin{center}\rule{0.5\linewidth}{\linethickness}\end{center}

\section{Many Labs 2 map}\label{many-labs-2-map}

.center{[}{]}

\begin{center}\rule{0.5\linewidth}{\linethickness}\end{center}

\section{Many Labs 2 Results}\label{many-labs-2-results}

.center{[}{]}

\begin{center}\rule{0.5\linewidth}{\linethickness}\end{center}

.center{[}{]}

\begin{center}\rule{0.5\linewidth}{\linethickness}\end{center}

\section{Many Labs 2 Results}\label{many-labs-2-results-1}

\begin{longtable}[]{@{}l@{}}
\toprule
\begin{minipage}[b]{0.03\columnwidth}\raggedright\strut
- 14/28 successful replications - p \textless{} .0001, non-trivial
effect size, same direction as original - One weakly supported, p = .03
but near-zero effect size\strut
\end{minipage}\tabularnewline
\midrule
\endhead
\begin{minipage}[t]{0.03\columnwidth}\raggedright\strut
- 75\% has smaller effect size than original - Median original d = 0.60
- Median replication d = 0.15\strut
\end{minipage}\tabularnewline
\bottomrule
\end{longtable}

\begin{itemize}
\tightlist
\item
  No evidence that the order of the studies mattered
\item
  In general, didn't matter if the study came first, last, or in any
  other position.
\item
  Same as ML1 and ML3
\end{itemize}

--

.center{[}{]}

\begin{center}\rule{0.5\linewidth}{\linethickness}\end{center}

.center{[}{]}

\begin{center}\rule{0.5\linewidth}{\linethickness}\end{center}

\section{Many Labs 2 Heterogeneity}\label{many-labs-2-heterogeneity}

\begin{longtable}[]{@{}r@{}}
\toprule
\begin{minipage}[b]{0.03\columnwidth}\raggedleft\strut
- Q statistic: (\textasciitilde{} significance test for variation across
sites exceeding chance) - 11/28 (39\%) showed significant heterogeneity
- Nearly all from larger-effects studies\strut
\end{minipage}\tabularnewline
\midrule
\endhead
\begin{minipage}[t]{0.03\columnwidth}\raggedleft\strut
- I²: - 36\% showed at least medium heterogeneity - Likely not an
appropriate measure in this case: - See \url{osf.io/frbuv} (Marcel van
Assen), \url{Datacolada.org/63}, Borenstein+Higgins\strut
\end{minipage}\tabularnewline
\bottomrule
\end{longtable}

\begin{itemize}
\tightlist
\item
  Tau is probably best
\item
  SD across samples in the unit of the effect size (after accounting for
  sampling error)
\end{itemize}

\begin{center}\rule{0.5\linewidth}{\linethickness}\end{center}

.center{[}{]}

\begin{center}\rule{0.5\linewidth}{\linethickness}\end{center}

.center{[}{]}

\begin{center}\rule{0.5\linewidth}{\linethickness}\end{center}

.center{[}{]}

\begin{center}\rule{0.5\linewidth}{\linethickness}\end{center}

\section{Discussion}\label{discussion}

\begin{itemize}
\item ~
  \subsection{Lack of sensitivity to
  sample/context}\label{lack-of-sensitivity-to-samplecontext}
\item ~
  \subsection{Not reasonable to discount replications by default based
  on
  sample}\label{not-reasonable-to-discount-replications-by-default-based-on-sample}
\item ~
  \subsection{Need to test moderators}\label{need-to-test-moderators}
\item ~
  \subsection{BUT: Mostly student samples, mostly short computerized
  studies}\label{but-mostly-student-samples-mostly-short-computerized-studies}
\item
  Replication rate aligns with others
\item ~
  \subsection{Meaningful?}\label{meaningful}
\item ~
  \subsection{Many studies replicate
  robustly}\label{many-studies-replicate-robustly}

  \subsection{-Overall takeaways
  (personally):}\label{overall-takeaways-personally}
\item ~
  \subsection{Robust replicability is a feasible goal (for many
  studies)}\label{robust-replicability-is-a-feasible-goal-for-many-studies}
\item ~
  \subsection{\texorpdfstring{Nudges me towards ``false-positive''
  explanation for replication failures (in
  general)}{Nudges me towards false-positive explanation for replication failures (in general)}}\label{nudges-me-towards-false-positive-explanation-for-replication-failures-in-general}
\item
  Reinforces need for preregistration/Registered Reports
\end{itemize}

\begin{center}\rule{0.5\linewidth}{\linethickness}\end{center}

class: center, middle

\section{Thanks!}\label{thanks}

Special thanks to co-leads Fred Hasselman, Michelangelo Vianello, and
Brian Nosek + 186 other co-authors.

Questions/comments?

~~~~~~


\end{document}
